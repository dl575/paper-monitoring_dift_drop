\section{Conclusion}
\label{sec:conclusion}

Parallel run-time monitoring techniques are attractive solutions for improving
the reliability, security, and debugging capabilities of systems. In this
paper, we have presented an architecture that enables adjustable overheads through partial monitoring. 
The architecture uses a dataflow tracking engine to prevent false positives
through tracking invalidation information. In addition, we show how the
architecture can be extended to enable null metadata filtering.
Given this architecture, we explore the design choices related to the dropping policy.
With partial monitoring, we see that with a 1.5x overhead target, BC can still
achieve 85\% average coverage and UMC can still achieve 14\% 
average coverage.
At this overhead target, IMP shows less than 2\% error on average.
LS can be pushed down to 1.1x overhead and still achieve 61\% average coverage.


% This is done by
% using a hardware dataflow tracking engine in order to track and filter out
% monitoring for null metadata. In addition, the dataflow engine tracks invalid
% metdata flows in order to enable partial monitoring without false positives.
% Our experimental results show that filtering can greatly reduce the average
% overheads of monitoring from 7x, 18x, and 11x for UMC, BC, and DIFT down to 4x
% for UMC and BC and just 13\% for DIFT. With partial monitoring, we see that
% with a 10\% overhead budget, BC can still achieve 82\% coverage and UMC can
% still achieve 32\% coverage. Finally, we show that although source-only dropping can
% achieve better coverage than unrestricted dropping when it works, it is not
% able to match overhead targets as well as unrestricted dropping.
