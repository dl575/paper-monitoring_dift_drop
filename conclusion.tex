\section{Conclusion}
\label{sec:conclusion}

Parallel run-time monitoring techniques are attractive solutions for improving
the reliability, security, and debugging capabilities of systems. In this
paper, we have presented an architecture that reduces the overheads of
monitoring through filtering and adjustable partial monitoring. This is done by
using a hardware dataflow tracking engine in order to track and filter out
monitoring for null metadata. In addition, the dataflow engine tracks invalid
metdata flows in order to enable partial monitoring without false positives.
Our experimental results show that filtering can greatly reduce the average
overheads of monitoring from 7x, 18x, and 11x for UMC, BC, and DIFT down to 4x
for UMC and BC and just 13\% for DIFT. With partial monitoring, we see that
with a 50\% overhead budget, BC can still achieve 83\% coverage and UMC can
still achieve 51\% coverage. Finally, we show that although source dropping can
achieve better coverage than unrestricted dropping when it works, it is not
able to match overhead targets as well as unrestricted dropping.
