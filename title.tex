\title{
Dataflow-Guided Filtering for Efficient and Adjustable Run-Time Monitoring
}

\ifanonymized{\author{Anonymized}}
{\author{Daniel Lo, Tao Chen, Mohamed Ismail, and G. Edward Suh}}

\date{}
\maketitle

\thispagestyle{empty}

\begin{abstract}

Recent studies have proposed various parallel run-time monitoring techniques to
improve the reliability, security, and debugging capabilities of computer
systems. These run-time monitors can introduce large overheads in performance
and energy to the system.  Traditionally, these overheads have been an
all-or-nothing cost; monitoring cannot be used if the overheads are considered
too large.  In this paper, we show how a hardware-based dataflow tracking
engine can be used to filter out unnecessary monitoring and to enable partial
monitoring.  We are able to filter out unnecessary monitoring related to clean
metadata in order to greatly reduce the overheads of monitoring. In addition,
we present a hardware architecture can further reduce overheads by enabling a
trade-off between the amount of monitoring performed and the overheads
incurred. In order to prevent false positives due to this partial monitoring,
we extend the dataflow engine to track dropped monitoring flows. We investigate
some of the trade-offs involved with partial monitoring in terms of how closely
an overhead target can be matched versus the coverage that can be achieved.
% Our experiments show that the system is able to
% closely meet the target overhead and achieve significant monitoring coverage
% for reasonable overhead targets.
In our experiments, we see that filtering is able to decrease monitoring
overheads from XX\%-XX\% on average, depending on the monitoring scheme, to
XX\%-XX\%. Additionally, we are able to still achieve XX\% coverage with a 10\%
overhead target.

\end{abstract}
