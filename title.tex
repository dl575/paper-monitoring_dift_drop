\title{
Dataflow-Guided Filtering for Efficient and Adjustable Run-Time Monitoring
}

\ifanonymized{\author{Anonymized}}
{\author{Daniel Lo, Tao Chen, Mohamed Ismail, and G. Edward Suh}}

\date{}
\maketitle

\thispagestyle{empty}

\begin{abstract}

Recent studies have proposed various parallel run-time monitoring techniques to
improve the reliability, security, and debugging capabilities of computer
systems. However, these run-time monitors can introduce large performance and energy
overheads, especially when performed on programmable cores.
%Traditionally, these overheads have been an
%all-or-nothing cost; monitoring cannot be used if the overheads are considered
%too large.  
In this paper, we introduce a hardware dataflow tracking engine that can be
used to filter out unnecessary monitoring and enable adjustable partial monitoring.
The dataflow engine can identify events with null metadata so that they can
be filtered out. To further reduce the overhead of monitoring, we propose
to enable a trade-off between monitoring coverage and overhead by dropping certain
monitoring operations. For this partial monitoring, the dataflow engine is used
to track dropped monitoring flows so that false positives can be avoided.
Given the architecture, we also investigate how the dropping decisions should be
made for partial monitoring and show that there exists a trade-off between closely
matching an overhead target and the coverage that can be achieved.
% Our experiments show that the system is able to
% closely meet the target overhead and achieve significant monitoring coverage
% for reasonable overhead targets.
The experimental results show that filtering can significantly reduce monitoring
overheads from XX\%-XX\% on average, depending on the monitoring scheme, to
XX\%-XX\%. Additionally, the partial monitoring is shown to still achieve XX\% coverage 
only with a 10\% overhead target.

\end{abstract}
