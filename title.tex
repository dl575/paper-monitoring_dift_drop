\title{
Dataflow-Guided Filtering for Efficient and Adjustable Run-Time Monitoring
}

\ifanonymized{\author{Anonymized}}
{\author{Daniel Lo, Tao Chen, Mohamed Ismail, and G. Edward Suh}}

\date{}
\maketitle

\thispagestyle{empty}

\begin{abstract}

Recent studies have proposed various parallel run-time monitoring techniques to
improve the reliability, security, and debugging capabilities of
computer systems. These run-time monitors introduce overheads in performance
and energy to the system. Traditionally, these overheads have been an
all-or-nothing cost; monitoring cannot be used if the overheads are considered too large.
% where if monitoring is enabled, then the program will incur, in full, the
% associated overheads. 
%That is, if the overheads are considered too large, then monitoring cannot be
%enabled. 
In this paper, we present a run-time monitoring architecture that
enables a trade-off between the coverage and the overheads.
The system allows an overhead or coverage target to be specified,
and limits the amount of monitoring performed at run-time to meet the target. 
However, simply skipping monitoring operations can lead to false
positives, signaling errors when none have occurred. Our proposed architecture
prevents false positives using a method that invalidates certain metadata
when monitoring is not performed. Dedicated hardware that performs dynamic
dataflow tracking is used to propagate this invalidation
information in order to prevent false positives. In addition, we can use this
dataflow hardware to also track the presence of initialized metadata. This
allows us to filter out events on uninitialized metadata in order to reduce
monitoring overheads.
% Our experiments show that the system is able to
% closely meet the target overhead and achieve significant monitoring coverage
% for reasonable overhead targets.
In our experiments, we see that the system is able to come within 2\% of the
specified execution time overhead in almost all cases.  In addition, with a
performance overhead target of 10\%, 49-90\% of monitoring coverage is achieved
on average depending on the monitoring technique.

\end{abstract}
