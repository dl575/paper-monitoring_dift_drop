\section{Filtering Null Metadata}
\label{sec:filtering}

One way to reduce the number of monitoring events that must be handled by the
monitor is to filter out monitoring events that correspond to operating on null
or uninitialized metadata. Null metadata typically corresponds to events that
are not relevant to the monitor. For example, Hardbound
\cite{hardbound-asplos08} filters out operations on non-pointer (i.e., no base
and bounds metadata) instructions since it is not relevant to array bounds
checking. More recently, FADE \cite{fade-hpca14} has been proposed a general
hardware module to perform this null metadata filtering for a variety of
monitoring schemes. 

We enable this null metadata filtering by
simply extending the dataflow flags to be two bits wide in order to keep track
of both validity due to dropping and null information.
All flags are initialized to valid and null. Whenever the monitor sets metadata
to a non-null value, it will also mark the corresponding flag in the dataflow
engine as non-null. In addition to checking validity when reading flags, the
flags are also checked for whether the metadata is null. An event only
continues to the monitor if all source flags are valid and any source flag is
marked as non-null.

