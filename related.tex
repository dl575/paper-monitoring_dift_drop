\section{Related Work}
\label{sec:related}

% Monitoring
Many monitoring schemes have been developed for various security, reliability,
debugging, and other capabilities.  As we described in
Section~\ref{sec:extensions}, there exist monitoring schemes for uninitialized
memory check \cite{mondrian-asplos02}, array bounds check
\cite{hardbound-asplos08, clause-ase07}, and dynamic information flow
tracking \cite{dift-asplos04, raksha-isca07, loki-osdi08}.
Memtracker \cite{memtracker-hpca07} uses monitoring to check for various
memory-related bugs and errors.  Control flow verification techniques perform
monitoring to check that program execution follows expected control flows
\cite{schuette-comp87, impres-dac06,
kayaalp-isca12}.  Our monitoring architecture is designed to
be generally applicable to any of these instruction-grained monitoring schemes.
In particular, our proposed architecture may be especially useful for some of
these more complex schemes which also have higher overheads.  Similarly,
although many of these schemes showed low overheads in custom hardware, our
architecture can be used to reduce the overheads when applying these schemes on
a multi-core platform.
 
% Filtering
FADE \cite{fade-hpca14} provides a more general filtering engine. It
filters out redundant updates and clean checks by reading metadata values. Our
dataflow engine does not require reading the actual metadata values which may
be large. We do not handle filtering redundant updates, but we do filter out
clean checks as well as clean (i.e., uninitialized) propagations. We are able
to do this because the dataflow engine can also propagate the uninitialized
information. In addition, our dataflow engine can be used to provide adjustable
overheads without false positives which is not a capability described by FADE.

% Sampling
Statistical sampling is a method that has been applied to various debugging
techniques in order to reduce their performance overheads. For example, Liblit
et al. sampled program runs of thousands of users in order to find bugs
\cite{liblit-pldi05}. This idea has also been applied to detecting memory
leaks~\cite{chilimbi-asplos04}, dataflow analysis~\cite{greathouse-cgo11}, and
other run-time monitoring techniques. The Testudo project
\cite{testudo-micro08} specifically targets hardware-based run-time monitors.
These sampling-based techniques all limit the debugging capabilities in order
to provide lower performance overheads, similar to our architecture's goals.
However, these approaches are not able to explicitly set an overhead budget, as
our system is able to do. In addition, with the exception of Testudo, these
methods have not considered parallel monitoring. Finally, we note that the
concept of distributing the monitoring over many users and runs to increase
coverage can also be enabled using our architecture.

% Limited Monitoring
In terms of controlling overheads, the Quality Virtual Machine (QVM) is a
modification of the Java Virtual Machine (JVM) that supports run-time
monitoring with controllable overheads \cite{qvm-oopsla08}. Similarly, Huang et
al. created a framework for controlling the overheads of software-based
monitoring \cite{huang-sttt12}. These projects are targeting a similar problem
to the one we address in this paper. However, both projects are focused on
software-based monitoring and thus their mechanisms modify the software
monitoring framework in order to enable or disable monitoring. Instead, our
architecture provides a general hardware mechanism to control overheads of
parallel monitoring architectures. In addition, these works prevented
false positives by enabling and disabling monitoring using a source-dropping
policy. Lo et al. \cite{lo-rtas14} developed a hardware architecture to limit
monitoring in hard real-time systems. Their architecture performs all-event 
dropping in order to guarantee strict deadlines. Our system allows for a
trade-off between choosing source-event dropping, all-event dropping, or min-event
dropping depending on the needs of the designer or user which has not been
explored by these previous works. 
