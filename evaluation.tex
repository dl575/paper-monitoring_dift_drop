\section{Evaluation}
\label{sec:evaluation}

\subsection{Experimental Setup}
\label{sec:evaluation.setup}

We implemented our dataflow-guided monitoring architecture by
modifying the ARM version of the gem5 simulator \cite{gem5} to support parallel
run-time monitoring. We implement the monitor as a software-based monitor
running on a parallel processor core, similar to LBA \cite{lba-asid06}.
We model the main and monitoring cores as
running at 2.5
GHz with 4-way set-associative private L1 I/D caches and a shared 8-way 2 MB L2
cache. This setup is similar to the Snapdragon 801 processor commonly found in
mobile systems. The dataflow engine uses a 1 kB cache for invalid and null flags.

In order to explore the generality of the architecture for
different monitors, we implemented five different monitors: uninitialized
memory check (UMC), array bounds check (BC), dynamic information flow
tracking (DIFT), instruction-mix profiling (IMP), and LockSet-based race detection (LS).  Uninitialized memory
check seeks to detect loading from
memory locations that are not initialized first. Array bounds check, as
described in Section~\ref{sec:dropping}, is a monitoring scheme that aims to
detect buffer overflows where memory accesses go beyond the boundaries of an
array. We modify the implementation of {\tt malloc} to set base and bound
metadata information. Dynamic information flow tracking is a security
monitoring scheme
which detects when information from untrusted sources is used to affect the
program control flow (i.e., indirect control instructions). For the benchmarks we consider, we mark data read from
files as untrusted. We implemented a multi-bit DIFT scheme which marks
untrusted data with a 32-bit metadata identifier so
that if an error is detected, it is possible to have information about where
the data originated from. Instruction-mix profiling counts the number of ALU,
load, store, and control instructions that are executed. This profiling
information can be useful for performing optimizations or to detect malicious
software \cite{tang-raid14}.
LockSet \cite{eraser-tocs97} attempts to detect possible race conditions in multi-threaded programs
by tracking metadata about which locks are protecting shared memory
locations. If a shared memory location is accessed while unprotected
then a race condition may exist. Since we implement the monitor using a
processor core and our dataflow engine was designed to be generally applicable,
the same hardware platform supports all of these monitoring techniques.

We tested our system using benchmarks from SPECint
CPU2006 \cite{spec2006}. Since our implementation of BC depends on the
modification of {\tt malloc} to set array bounds information, we focus on the C
SPECint benchmarks. Although we do not
show results for the C++ benchmarks, we note that the results for UMC, DIFT, and IMP
for these benchmarks are similar to the other results shown.
We simulated each benchmark for 2 billion instructions. 
% An initial slack of 2
% million cycles, which is less than 1\% of
% the total execution time, is given for a 1.1x target overhead. This initial
% slack is scaled proportionally for other overhead targets. 

Since LockSet race detection requires multi-threaded programs, we tested it using
the applications from the SPLASH-2 benchmark set \cite{splash-isca95}. We ran these benchmarks for
200 million instructions eachi using 2 main cores to run the benchmark application.
Each main core was connected to a dataflow engine and a monitoring core.

\subsection{Baseline Monitoring Overheads}

% % Full monitoring overheads
% \begin{figure}
%   \begin{center}
%     \includegraphics[width=\columnwidth]{figs/data_full_mon.pdf}
%     \vspace{-0.2in}
%     \caption{Full monitoring overheads.}
%     \label{fig:evaluation.full_mon}
%     \vspace{-0.1in}
%   \end{center}
% \end{figure} 

% Figure~\ref{fig:evaluation.full_mon} shows the execution times of
% performing full monitoring normalized to the execution times of the benchmarks
% without monitoring. In these results, no filtering or partial monitoring is
% done. 
% Our initial results for performing full core-based monitoring show severalfold slowdowns.
% UMC shows normalized execution times from 5x to 16x with an average
% of 10x. BC shows normalized execution times of 8-24x with an average of
% 16x while DIFT shows normalized execution times of 4x-14x with an average of
% 9x. IMP shows normalized execution times from 2-6x with an average of 4x and LS
% shows slowdowns of 1.8-3.3x with an average of 2.4x. 

% Filter monitoring overheads
\begin{figure}
  \begin{center}
    \includegraphics[width=\columnwidth, clip=true, trim=0 0.5in 0 0.1in]{figs/data_filtering.pdf}
    \vspace{-0.2in}
    \caption{Monitoring overheads with null metadata filtering.}
    \label{fig:evaluation.filtering}
    \vspace{-0.1in}
  \end{center}
\end{figure}

Figure~\ref{fig:evaluation.filtering} shows the
execution times of performing monitoring with null filtering enabled normalized
to the execution time of the benchmarks without monitoring. UMC sees normalized
execution times of 1.7-7x with an average of 4x.
For BC, normalized execution
are 3x on average and range from 1.1x to 8.3x.
% This is not surprising since in the baseline
% implementation all loads and stores needed to be monitored. However, with
% filtering, only loads and stores corresponding to arrays need to be forwarded.
DIFT sees an average of 1.1x slowdown with a maximum slowdown of 1.3x with null
metadata filtering. This low overhead is due, in part, to the fact
that for our implementation of DIFT on SPEC
benchmarks, we only mark data read from files as tainted. Instead, if we
targeted network or streaming applications, which have larger amounts of
untrusted input data, we would observe higher overheads. IMP shows normalized
execution times of 2-6x with an average of 4x. Overheads for LS (not pictured)
applied to the SPLASH-2 benchmarks are
1.03-1.72x after null metadata filtering and the average
overhead observed is 1.33x.
Our baseline system and overheads are similar to previous multi-core
monitoring systems such as LBA-accelerated \cite{lba-isca08} and FADE
\cite{fade-hpca14} (see
Table~\ref{tab:monitoring.previous_overheads}). In
Section~\ref{sec:evaluation.fpga}, we also evaluate a higher performance, FPGA-based
monitor that shows low tens of percent of overhead. 

\subsection{Coverage with Adjustable Partial Monitoring}

% BC sweep
\begin{figure}
  \begin{center}
    \includegraphics[width=\linewidth, clip=true, trim=0 0.3in 0 0.1in]{figs/data_bc_sweep.pdf}
    \vspace{-0.2in}
    \caption{Coverage vs. varying overhead budget for BC.}
    \label{fig:evaluation.bc_sweep}
    \vspace{-0.2in}
  \end{center}
\end{figure}

% UMC sweep
\begin{figure}
  \begin{center}
    \includegraphics[width=\linewidth, clip=true, trim=0 0.3in 0 0.1in]{figs/data_umc_sweep.pdf}
    \vspace{-0.2in}
    \caption{Coverage vs. varying overhead budget for UMC.}
    \label{fig:evaluation.umc_sweep}
    \vspace{-0.1in}
  \end{center}
\end{figure}

% LS sweep
\begin{figure}
  \begin{center}
    \includegraphics[width=\linewidth, clip=true, trim=0 0.3in 0 0.1in]{figs/data_ls_sweep.pdf}
    \vspace{-0.2in}
    \caption{Coverage vs. varying overhead budget for LS.}
    \label{fig:evaluation.ls_sweep}
    \vspace{-0.1in}
  \end{center}
\end{figure}

% IMP sweep
\begin{figure*}
  \begin{center}
    \includegraphics[width=\linewidth, clip=true, trim=0 0.3in 0 0.1in]{figs/data_imp_sweep.pdf}
    \vspace{-0.2in}
    \caption{Average error vs. varying overhead budget for instruction-mix profiling. Whisker bars show min and max errors.}
    \label{fig:evaluation.imp_sweep}
    \vspace{-0.1in}
  \end{center}
\end{figure*}

In this section, we evaluate the effectiveness of
using partial monitoring to trade-off coverage for reduced overheads.
For these results, we use a slack-based, unrestricted dropping policy.
Figure~\ref{fig:evaluation.bc_sweep} shows the monitoring coverage achieved by
array bounds check as we vary the overhead budget. 
We define \emph{monitoring coverage} as the 
percentage of dynamic checks that are performed 
(indirect jumps in DIFT, loads in UMC, and memory accesses in BC and LS). 
The metric is chosen to understand
how likely an error/attack instance is to be detected on an individual system, 
assuming that errors/attacks are uniformly distributed across checks. 
%While we could not evaluate real errors/attacks due to difficulty in setting up
%a large number of bugs and exploits, we believe that the percentage of checks
%provides a good estimate of detection probability when errors/attacks are 
%uniformly distributed across checks.
Note that this metric is different from the
percentage of monitoring events that are not dropped, which includes
non-check instructions.
The bottom portion of each
bar shows the coverage for a 1.1x overhead target and the additional
coverage for increased overhead targets are stacked above this.

We see that by varying the overhead budget, the coverage achieved also varies.
With only a 1.1x overhead target (10\% slowdown), array bounds check still
achieves 84\% monitoring coverage on average. The high coverage
achieved with such low overheads is due to two main effects.  The first is that
monitoring can be done in parallel, providing a certain level of monitoring coverage without
introducing overheads. The second effect is that there may still exist a large number of
monitoring events that do not lead to security or reliability checks. As a
result, dropping these events can reduce overheads without a large impact on
monitoring coverage.

Figure~\ref{fig:evaluation.umc_sweep} shows the analogous graph for UMC. Again
we see that varying overhead budgets enables partial monitoring. We see that
with a 1.1x overhead target, UMC achieves 25\% monitoring coverage on average.
By increasing this overhead budget to 1.5x, UMC achieves 43\% coverage on
average. Even higher coverage can be achieved by allowing higher overheads.
Similarly, Figure~\ref{fig:evaluation.ls_sweep} shows the coverage achieved by LS.
With a 1.1x overhead, LS achieves 53\% coverage on average. This increases to
71\% coverage with a 1.2x overhead. 
We do not show a graph for DIFT because the overhead target of 1.1x is enough for
all benchmark tested to reach 100\% coverage.
%Since the overheads of DIFT are low to begin with, setting an overhead target
%1.1x is enough for all benchmarks tested to reach 100\% coverage.

The instruction-mix profiling monitor does not perform check operations. Thus,
the concept of coverage is not applicable here. Instead, for each instruction
type profiled, we calculate its count as a percentage of the total instructions
monitored. We take the difference of these percentages compared to the case
when full monitoring
is performed to calculate the error. Figure~\ref{fig:evaluation.imp_sweep}
shows the min, max, and average error across
instruction types for each benchmark and for varying overheads. We see that
with a 1.1x overhead, a max error of 9\% is observed across the benchmarks and the
average error across instruction types and benchmarks is less than 2\%. 

\subsection{FPGA-based Monitor}
\label{sec:evaluation.fpga}

% Flex UMC sweep
\begin{figure}
  \begin{center}
    \vspace{-0.1in}
    \includegraphics[width=\linewidth, clip=true, trim=0 0.3in 0 0.1in]{figs/data_fpga_umc_sweep.pdf}
    \vspace{-0.2in}
    \caption{Coverage versus varying overhead budget for UMC running on an FPGA-based monitor.}
    \label{fig:evaluation.fpga_umc_sweep}
    \vspace{-0.1in}
  \end{center}
\end{figure}

In addition to evaluating our system for a core-based monitor, we also
evaluated partial monitoring on a higher performance FPGA-based monitor. This
setup is based on the FlexCore \cite{flexcore-micro10} setup and uses a
fully-pipelined monitor running on an FPGA fabric at half the frequency of the
main core. We only show results for UMC due to space constraints though other
monitors show similar trends. In this case, the full overheads of
monitoring range from 1.1-1.7x. Figure~\ref{fig:evaluation.fpga_umc_sweep} shows
the coverage achieved as we sweep the overhead target from 1.01-1.7x. We see that partial
monitoring can allow the overheads of such a system to be pushed below 10\%
while still providing over 50\% coverage.

\subsection{Comparing Dropping Policies}

% % UMC exec time
% \begin{figure}
%   \begin{center}
%     \includegraphics[width=\columnwidth]{figs/data_umc_exec_time.pdf}
%     \vspace{-0.2in}
%     \caption{Error in meeting overhead budget for uninitialized memory check for different dropping policies.}
%     \label{fig:evaluation.umc_exec_time}
%     \vspace{-0.2in}
%   \end{center}
% \end{figure}
% 
% % UMC coverage across policies
% \begin{figure}
%   \begin{center}
%     \includegraphics[width=\columnwidth]{figs/data_umc_coverage.pdf}
%     \vspace{-0.2in}
%     \caption{Coverage for uninitialized memory check for different dropping policies.}
%     \label{fig:evaluation.umc_coverage}
%     \vspace{-0.1in}
%   \end{center}
% \end{figure}

\begin{figure}
  \begin{center}
  \vspace{-0.1in}
  \subfloat[Overhead budget error.]{
    \includegraphics[width=\columnwidth, clip=true, trim=0 0.3in 0 0.1in]{figs/data_umc_exec_time.pdf}
    \label{fig:evaluation.umc_exec_time}
  }
  \vspace{-0.1in}
  \subfloat[Coverage.]{
    \includegraphics[width=\columnwidth, clip=true, trim=0 0.3in 0 0.1in]{figs/data_umc_coverage.pdf}
    \label{fig:evaluation.umc_coverage}
  }
  \vspace{-0.1in}
  \caption{Comparing dropping policies for UMC.}
  \vspace{-0.2in}
  \end{center}
  \label{fig:evaluation.umc_policies}
\end{figure}

In this section we evaluate the trade-offs between unrestricted dropping and source-only dropping.
The results are shown for an overhead target of 1.5x.
Figure~\ref{fig:evaluation.umc_exec_time} shows the difference between the
overhead budget and the run-time monitoring overheads for UMC. A positive value
means that the overhead target was overshot while a negative value indicates
that the overhead budget was met. We see similar results for unrestricted
dropping and source-only dropping due to the fact that UMC consists of a large
number of short, independent monitoring chains.
Figure~\ref{fig:evaluation.umc_coverage} shows the coverage of UMC for
unrestricted dropping and source-only dropping. We see that, in several cases, source-only dropping
achieves higher coverage than unrestricted dropping. On average, unrestricted
dropping achieves 46\% coverage while source-only dropping achieves 49\%
coverage. This is due
to the fact that some of the overheads of monitoring for unrestricted dropping
are being spent on wasted work as discussed in Section~\ref{sec:policies.which}.

% % BC exec time
% \begin{figure}
%   \begin{center}
%     \includegraphics[width=\columnwidth]{figs/data_bc_exec_time.pdf}
%     \vspace{-0.2in}
%     \caption{Error in meeting overhead budget for array bounds check for different dropping policies.}
%     \label{fig:evaluation.bc_exec_time}
%     \vspace{-0.2in}
%   \end{center}
% \end{figure}
% 
% % BC coverage across policies
% \begin{figure}
%   \begin{center}
%     \includegraphics[width=\columnwidth]{figs/data_bc_coverage.pdf}
%     \vspace{-0.2in}
%     \caption{Coverage for array bounds check for different dropping policies.}
%     \label{fig:evaluation.bc_coverage}
%     \vspace{-0.2in}
%   \end{center}
% \end{figure}

\begin{figure}
  \begin{center}
  \vspace{-0.1in}
  \subfloat[Overhead budget error.]{
    \includegraphics[width=\columnwidth, clip=true, trim=0 0.3in 0 0.1in]{figs/data_bc_exec_time.pdf}
    \label{fig:evaluation.bc_exec_time}
  }
  \vspace{-0.1in}
  \subfloat[Coverage.]{
    \includegraphics[width=\columnwidth, clip=true, trim=0 0.3in 0 0.1in]{figs/data_bc_coverage.pdf}
    \label{fig:evaluation.bc_coverage}
  }
  \vspace{-0.1in}
  \caption{Comparing dropping policies for BC.}
  \vspace{-0.2in}
  \end{center}
  \label{fig:evaluation.bc_policies}
\end{figure}

Next, we evaluate these trade-offs between source-only dropping and unrestricted dropping for BC.
Figure~\ref{fig:evaluation.bc_exec_time} shows the overhead differences for BC
and Figure~\ref{fig:evaluation.bc_coverage} shows the coverage for BC.
From Figure~\ref{fig:evaluation.bc_exec_time}, we see that for several
benchmarks, source-only dropping fails to meet the specified overhead target.
The overshoot of the overhead target is quite high with overhead differences
of 420\% for {\tt bzip2}, 120\% for {\tt gcc}, and 1500\% for {\tt hmmer}.
Since only infrequently occurring array allocations are considered as source events for BC, it can be 
difficult for source-only dropping to match overhead targets. Although
Figure~\ref{fig:evaluation.bc_coverage} shows higher coverage for
source-only dropping, this is largely due to the fact that is running with
higher overheads.

From these results, we see that for monitors with many short, independent
chains, such as UMC, source-only dropping can provide better coverage than
unrestricted dropping with the same overhead. For other monitors, such as BC,
source-only dropping is prone to overshoot the overhead target.
Thus, for applications where staying within an overhead target is especially
important, such as soft real-time systems, a slack-based, unrestricted dropping
policy is more appropriate. On the other hand, if maximum coverage is desired
and occasional slowdowns are acceptable, then source-only dropping can provide
better coverage on average.

\subsection{Multiple-Run Coverage}

% Multi-run coverage with unrestricted dropping
\begin{figure}
  \begin{center}
    \includegraphics[width=\columnwidth, clip=true, trim=0 0.3in 0 0.1in]{figs/data_multirun_coverage.pdf}
    \vspace{-0.2in}
    \caption{Coverage over multiple runs for BC with a 10\% probability to not drop events.}
    \label{fig:evaluation.multirun}
    \vspace{-0.1in}
  \end{center}
\end{figure}

One application of partial monitoring with low overheads is to enable
cooperative debugging. 
The idea with cooperative debugging is to use partial
monitoring with very low overheads across a large number of users or runs. By
varying the pattern of partial monitoring done on each run, the goal is to
achieve high coverage across multiple runs. Varying the monitoring that is
done for different runs can be achieved by using a probabilistic dropping policy.
Figure~\ref{fig:evaluation.multirun} shows the total
coverage achieved over multiple runs for array bounds check using
unrestricted and source-only dropping policies with probabilistic dropping.
These numbers are averaged across all benchmarks. Here, we use a 10\%
probability
that events are not dropped. Since the goal is code coverage, coverage here is
measured as the percentage of static instructions which are monitored in at
least one of the runs. Each run
was simulated for 200 million instructions. 
We see that for the unrestricted dropping policy there is almost no increase in
coverage.  This is due to the fact that it is likely that at least one
monitoring event in a metadata dependence chain will be dropped with the unrestricted dropping
policy. 
Instead, source-only
dropping is much better suited for achieving high coverage over multiple runs.
Source-only dropping shows a 5\% increase in coverage with only seven runs.
Over many runs, source-only dropping should be able to
reach 100\% code coverage.

\subsection{Area and Power Overheads}

% Area and Power Overheads
\begin{table}[tb]
  \begin{center}
    \vspace{-0.0in}
    \begin{footnotesize}
    
% Full monitoring at zero slack

\begin{tabular}{|c|c|c|}
\hline

{\bf Monitor} & {\bf Peak Power [mW]} & {\bf Runtime Power [mW]} \\ \hline\hline

UMC  & 47 (4.9\%) & 30 (7.7\%) \\ \hline
BC   & 69 (7.1\%) & 41 (10.7\%) \\ \hline
DIFT & 72 (7.4\%) & 41 (10.6\%) \\ \hline
IMP  & 41 (4.3\%) & 20 (5.1\%) \\ \hline
LS   & 31 (3.2\%) & 13 (3.9\%) \\ \hline

\end{tabular}

    \end{footnotesize}
    \caption{Average power overhead for dropping hardware. Percentages 
    are normalized to the main core power.}
    \vspace{-0.2in}
    \label{tab:evaluation.area_power}
  \end{center}
\end{table}

Adding the dataflow engine in order to enable filtering and partial monitoring adds
overheads in terms of area and power. We use McPat \cite{mcpat-micro09} to get
a first-order estimate of these area and power overheads in a 40 nm technology
node. McPat estimates the main core area as 2.71 mm$^2$ and the peak power usage as
965 mW averaged across all benchmarks. The average runtime power usage was 385 
mW. These area and power numbers consist of the core and
L1 cache, but do not include L2 cache, memory controller, and other
peripherals. The power numbers include dynamic as well as static (leakage)
power. For the dataflow engine, we modeled the ALUs used for address
calculation, the dataflow flag register file and cache, the configuration
tables, and the filter decision table. These were modeled using the
corresponding memory and ALU objects in McPat. We
note that this is only a rough area and power estimate since components such as the
wires connecting these modules have not been modeled. However, this gives a
sense of the order-of-magnitude overheads involved with implementing our
approach.

Our results show that an additional 0.197 mm$^2$ of silicon area is needed, an
increase of 7\% of the main core area. Table~\ref{tab:evaluation.area_power}
shows the peak and runtime power overheads averaged across all benchmarks running with
a 1.5x monitoring overhead target. The peak power is 31-72 mW, which is
less than 8\% of the main core's peak power usage. Similarly, the average runtime power is 13-43
mW, corresponding to 4-11\% of the main core's runtime power.

