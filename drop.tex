\section{Enabling Coverage/Overhead Trade-off}
\label{sec:drop}

Figure~\ref{dummy} shows the overheads of array bounds check with filtering
based on initialized metadata (see Section~\ref{sec:evaluation} for details on
the experimental methodology). The overheads range from XX\% to XX\%. In some
contexts, this overhead may still be considered to be too high. In this
section, we discuss how we can selectively perform monitoring in order to
match a specified overhead target.

The high-level idea is to drop monitoring operations in order to remain within
the specified overhead budget. Given an overhead budget, the system estimates
the run-time overhead caused by monitoring (Section~\ref{sec:drop.slack}). When
this overhead exceeds the budget, then the system drops monitoring operations.
In order to prevent false positives, we must be able to identify flows that
have been dropped. This can be done by extending the dataflow engine to keep
track of an additional ``drop'' flag (Section~\ref{sec:drop.false_positives}). 

\subsection{Measuring Overheads}
\label{sec:drop.slack}

In order to attempt to match an overhead budget, we need a way to estimate the
actual overheads incurred due to monitoring. More precisely, what we care about
is the difference between the actual overheads and the overhead budget.
Slack is essentially a measure of this overhead error. We can measure slack by
doing the following. 

% Slack
\begin{figure}
  \begin{center}
    \includegraphics[width=\columnwidth]{figs/slack.pdf}
    \vspace{-0.3in}
    \caption{Slack and its effect on monitoring over time.}
    \label{fig:drop.slack}
    \vspace{-0.2in}
  \end{center}
\end{figure}

% Slack tracking module
\begin{figure}
  \begin{center}
    \includegraphics[width=\columnwidth]{figs/stm.pdf}
    \vspace{-0.3in}
    \caption{Slack tracking and drop decision hardware.}
    \label{fig:drop.stm}
    \vspace{-0.1in}
  \end{center}
\end{figure}

In order to decide whether a monitoring event should be dropped or not, we need
a way to keep track of overheads. In this section, we present a specific method
for measuring execution time overheads. However, we note that other methods
could be used in order to target budgets on IPC, power, energy, etc.  For
example, one way to measure execution time overheads would be to insert
checkpoints in the program. By recording the execution time of these
checkpoints without monitoring, the overheads incurred while running with
monitoring could be determined. Although this would give an accurate
measurement of overheads, it would require modifying the program binary. In
addition, the overhead budget for monitoring would only be updated at
checkpoints.

Instead, we develop a model to estimate the execution time overhead budget in a
gradual manner and without the need to modify the main program. Specifically,
our model of execution time budget is that a percentage of the main program's
executed cycles is the number of cycles of overhead that we allow for
monitoring.  We define \emph{slack} as the number of cycles of monitoring
overhead that can be incurred while staying within the budget target.  For
example, if no monitoring overheads occur during 1000 cycles of the main
program's execution and the designer sets a 20\% overhead target, then the
slack that is built up during this period is 200 cycles. This slack can be used
to perform monitoring without exceeding the overhead budget.
Figure~\ref{fig:drop.slack} shows an example of how slack can change over time.
As monitoring overheads occur, the slack is decremented. If the slack falls
below the overhead expected for handling a monitoring event, then the event is
dropped. In addition to this accumulated slack, a small amount of initial slack
can be given in order for monitoring to be performed at the start of a program.

Figure~\ref{fig:drop.stm} shows a hardware slack tracking module (STM) for
keeping track of slack. The STM uses a counter that increments on every $k$-th
cycle of the main core. This $k$ can be calculated by taking the reciprocal of
the target budget. For example, if the target budget is 20\%, then the counter
increments on every 5th clock cycle.  The value of this counter is the
accumulated slack.  Whenever the main core is stalled due to the monitor, slack
is decremented.

In addition to keeping track of overheads, a policy must be developed to
determine when monitoring event should be dropped. Complex policies involving
multiple conditions and safety factors could be used to finely tune
the trade off between monitoring and performance overhead. Similarly, introducing
randomness into the policy could allow for better sampling in a cooperative
debugging application of monitoring. Here, we present one simple
dropping policy that is based on the accumulated slack.
When the monitor is initialized, the monitor will load
its needed slack in order to perform a monitoring operation into a register. When
there is a monitoring event in the FIFO to be processed, a hardware comparator
checks whether the accumulated slack is greater than or equal to the necessary
slack for full monitoring. If enough slack exists, the monitor is signaled to
perform the monitoring operation. Otherwise, the monitoring event is dropped.

\subsection{Preventing False Positives}
\label{sec:drop.false_positives}

Dropping a monitoring operation implies that some functionality of the monitor
has been lost. This may cause either false negatives, where an error that
occurs in the main program's execution is not detected, or false positives,
where the monitor incorrectly believes an error has occurred. For example, a
false positive can occur for UMC if a store monitoring event is dropped. This
causes the store memory location to not be marked as initialized. A subsequent
load for the memory location will incorrectly cause an error to be raised.  We
accept false negatives as the loss in coverage that we trade off in order to
limit overheads.  However, we must safely drop monitoring events in such a way
as to avoid false positives so that the system does not incorrectly raise an
error.

Dropping monitoring events can create false negatives and false positives.
False negatives are tolerated. We handle false positives by invalidating
metadata.

% Flag propagation rules for BC
\begin{table}[tb]
  \begin{center}
    \begin{small}
    
% Initalized/Invalid flag propagation rules for BC/DIFT

\begin{tabular}{|c||c|c|c|c||c|c|c|c|}
\hline

{\bf Instruction} & \multicolumn{4}{c||}{\bf Clean (src0, src1)} & \multicolumn{4}{c|}{\bf Invalid (src0, src1)} \\ \cline{2-9}
{\bf Type} & {\bf 00} & {\bf 01} & {\bf 10} & {\bf 11} & {\bf 00} & {\bf 01} & {\bf 10} & {\bf 11} \\ \hline\hline

{\tt load} & - & - & 1 & 1 & - & - & 1 & 1 \\ \hline
{\tt store} & - & - & 1 & 1 & - & - & 1 & 1 \\ \hline
{\tt ALU} & - & - & - & 1 & - & 1 & 1 & 1 \\ \hline

\end{tabular}

    \end{small}
    \caption{Flag propagation rules for array bounds check. X indicates that no
    propagation is done while 0 or 1 indicates that the destination flag is set
    to that value.}
    \label{tab:drop.propagation_rules}
    \vspace{-0.2in}
  \end{center}
\end{table}

Since metadata derived from dropped metadata is also invalid. We propagate the
drop flag in a similar manner to the initialization flags using the dataflow
engine. Since certain metadata is invalid, we can filter out useless monitoring
done on this invalid metadata. In the typical case, the propagation rule is
that if any of the source operands is marked as dropped, then the destination
operand is marked as dropped. This is slightly different that the
initialization flags which propagate an uninitialized state only if both
sources are uninitialized. Table~\ref{tab:drop.propagation_rules} shows the
propagation rules that we use for array bounds check.  However, we note that
the propagation rules can be customized depending on the monitor by
reconfiguring the Filter Decision Table.

\subsection{Dropping Decision}

% Varying slack's impact on coverage for UMC
\begin{figure}
  \begin{center}
    \includegraphics[width=\columnwidth]{figs/dataflow_graph.pdf}
    \vspace{-0.3in}
    \caption{Example dataflow graph for metadata.}
    \label{fig:drop.dataflow_graph}
    \vspace{-0.1in}
  \end{center}
\end{figure}

One of the issues with waiting until the overhead budget is reached (i.e.,
slack is zero) to start dropping monitoring events is that this can result in
wasted work. For example, consider the dataflow graph shown in
Figure~\ref{fig:drop.dataflow_graph}. If event {\tt E} is dropped, then this
means that monitoring for events {\tt C} and {\tt D} was wasted.

One method to eliminate this wasted work is to only make dropping decisions at
the root of these dataflow graphs. That is, we will decide to either monitor or
not monitor an entire metadata flow. We refer to this dropping decision policy
as \emph{source dropping} and we refer to the previous policy of waiting until
the overhead budget is met as \emph{last-minute dropping}. Source dropping will
result in no wasted work. However, because of the coarser-grained decision, it
may be more difficult to closely match overheads. Source dropping is expected
to work well when there are a large number of independent dataflows.

Another possible method is to make the dropping decisions somewhere in between
\emph{source dropping} and \emph{last-minute dropping}. From
Figure~\ref{fig:drop.dataflow_graph}, what we would like to do is make a
decision at event {\tt C} whether to perform monitoring on that branch of the
dataflow. If we choose to perform monitoring, then we want to perform all
monitoring on that flow. Instead, if we choose to drop event {\tt C}, then the
entire flow is dropped with no wasted work. Similarly, dropping event {\tt F}
drops the bottom flow with no wasted work and event {\tt A} can be dropped to
drop the entire shown dataflow with no wasted work. If the program can be
analyzed or profiled to identify these instructions which, if dropped, lead to
no wasted work, then this information can be used at run-time to minimize the
amount of wasted work. We refer to this policy as \emph{min-point dropping}.
Min-point dropping is a coarser-grained decision that last-minute dropping, but
finer-grained that source dropping. Thus, its ability to match overheads is
expected to sit between these other two policies. Similarly, there is still a
chance of wasted work due to strange splitting and merging dataflow graph
shapes but typically we expect much less wasted work than last-minute
dropping. Thus, we expect the coverage achieved by min-point dropping for a
certain overhead to fall between source and last-minute dropping. These three
dropping policies create a trade-off space between coverage achieved and
ability to match an overhead target.

\subsection{Multiple-Run Coverage}
Coverage can be considered for a single or average run instance. It can also be
measured over multiple users/runs in order to capture total bug coverage of a
program.

The system presented so far operates in a deterministic manner and so each run
will perform the same monitoring checks. However, we can introduce randomness
in order to achieve different dropping patterns over runs and achieve higher
multiple-run coverage.

\subsection{Multi-Core Implementation}

This can be extended to mulit-core by using multiple dataflow engines and a
similar coherency protocol as that used for data caches.
