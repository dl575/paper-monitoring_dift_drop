\section{Adjustable Run-Time Monitoring}
\label{sec:monitoring}

% Previous work overheads
\begin{table*}[t]
  \begin{center}
    \vspace{-0.0in}
    \begin{footnotesize}
    \begin{tabular}{|l|p{2in}|r|r|}

\hline
{\bf Name} & {\bf Description} & {\bf Average Overhead} & {\bf Worst-case Overhead} \\ \hline\hline

DIFT \cite{dift-asplos04} & Dedicated hardware for information flow tracking & 1.1\% & 23\% \\ \hline
FlexiTaint \cite{flexitaint-hpca08} & Dedicated hardware for configurable information flow tracking & 1.1\%-3.7\% & 8.7\% \\ \hline
Hardbound \cite{hardbound-asplos08} & Dedicated hardware for array bounds check & 5\%-9\% & 22\% \\ \hline
Harmoni \cite{harmoni-dsn12} & Configurable hardware for tag-based monitoring & 1\%-10\% & 20\% \\ \hline\hline

FlexCore \cite{flexcore-micro10} & FPGA-based monitor & 1.05x-1.44x & 1.84x \\ \hline
Log-Based Architecture \cite{lba-asid06} & Core-based monitor & 3.23x-7.80x & 11x \\ \hline
Log-Based Architecture Optimized \cite{lba-isca08} & Core-based monitor with hardware acceleration  & 1.02x-3.27x & 5x \\ \hline
Multi-core DIFT \cite{nagarajan-interact08} & Core-based with per-application optimized dynamic information flow tracking monitor & 1.48x & 2.2x \\ \hline\hline

LIFT \cite{lift-micro06} & Dynamic binary instrumentation for information flow tracking & 3.6x & 7.9x \\ \hline
Purfiy \cite{purify-usenix92} & Dynamic binary instrumentation for memory leaks & 2.3x & 5.5x \\ \hline
TaintCheck \cite{taintcheck-05} & Dynamic binary instrumentation for taint analysis & 10x & 27x \\ \hline


\end{tabular}

    \end{footnotesize}
    \caption{Trade-off between performance overhead and flexibility/complexity of run-time monitoring systems.}
    \vspace{-0.2in}
    \label{tab:monitoring.previous_overheads}
  \end{center}
\end{table*}

\subsection{Overhead of Run-Time Monitoring}

There have been a number of proposals for run-time monitoring systems exploring various
design points. % in the trade-off space between efficiency and flexibility.
Table~\ref{tab:monitoring.previous_overheads} summarizes some of representative designs
and their reported performance overhead. The previous studies clearly show that there
exist trade-offs among efficiency, flexibility, and hardware costs. 
For example, a run-time monitoring scheme can often be realized with fairly low
performance overhead (less than 20\%) if implemented with custom hardware that is
designed only for one monitor or a narrow set of monitors. However, the custom
hardware monitors cannot be modified or updated, and require dedicated silicon area. 
One the other hand, flexible systems that support a wide range of monitors lead 
to noticeable performance overhead, often too high for wide deployment in practice.
Software-only implementations \cite{FIXME} or multi-core monitors with minimal
hardware changes \cite{lba-asid06} are reported to have severalfold slowdowns.
On-chip FPGA monitors \cite{flexcore-micro10} and cores with monitoring accelerators
\cite{lba-isca08, fade-hpca14} can reduce overhead significantly, but still show
slowdowns of several tens of percents in some cases.
In today's monitoring systems, the overhead is often unpredictable because they
depend heavily on the characteristics of applications and monitoring operations.
In summary, users need to either pay noticeable overhead or the cost of custom
hardware in order to use fine-grained run-time monitoring in deployed systems.

\subsection{Partial Monitoring for Adjustable Overhead}

In this paper, we aim to develop a general framework that enables monitoring
overhead to be dynamically adjusted by dropping a portion of monitoring 
operations if necessary. In essence, this framework adds a new dimension to
the monitor design space by allowing coverage or accuracy to be traded off
for lower overhead. For example, the capability to adjust overhead allows users
to use monitoring with partial coverage in order to reduce performance
overhead. Alternatively, partial monitoring can allow users to use simple
one may choose to use simple hardware such as a
multi-core without elaborate acceleration features to provide partial monitoring
with the performance overhead that is comparable to custom hardware.

In this framework, a user can specify how much monitoring should be done such as a target 
overhead budget, a target coverage, a percentage of monitoring operations to be performed, etc.
Then, the framework dynamically drops a portion of monitoring operations to match
the target. In particular, this paper focuses on matching a performance overhead target
while maximizing the coverage. Given that the overhead of custom hardware monitors can
already be quite low, the focus will be enabling the trade-off in {\em general-purpose} 
monitoring systems that support a wide range of monitors.
We also consider the target overhead as a soft constraint and do not aim to
provide a strict worst-case guarantee.

\subsection{Applications and Metrics}

While it is ideal if run-time monitoring can be performed in full, we believe that
the capability to trade-off coverage/accuracy for lower performance/hardware overhead
will be useful in many application scenarios where full monitoring is not a viable option.
Here, we briefly discuss example applications of partial monitoring and the metrics
that are important in each case.

{\bf Cooperative testing, debugging, and protection:}
Recently studies \cite{FIXME} have shown that software testing, debugging,
or attack detection may be done in a cooperative fashion across
a large number of systems. In this case, each system is only willing to pay very low
overhead, say a few percents, and only performs a small subset of checks. 
A high coverage is achieved by having different systems check different parts of a program.
The partial monitoring framework allows high-overhead monitoring to be used in a
cooperative fashion. 
The main metric that represents the effectiveness of monitoring in this case is
the coverage (the percentage of checks there were performed) over multiple runs.

{\bf Monitoring of soft real-time systems:}
Soft real-time systems or interactive systems often need to meet deadlines or response
time requirements. Unfortunately, the overhead of run-time monitoring is often 
unpredictable and varies significantly depending on the application and monitoring
characteristics. The partial monitoring framework allows monitoring to be performed
while providing a level of guarantee on its impact on the execution time. 
In this case, it is important that the system can closely match the desired
overhead target while maximizing the effectiveness of monitoring.

{\bf Partial protection for low overhead:}
Even without real-time constraints, systems may have tight budgets for performance
or hardware overhead that they can tolerate for run-time monitoring. In such cases,
full monitoring cannot be enabled unless its overhead is low enough. The adjustable
monitoring allows partial protection on such systems. For example, array bounds
may be checked for a portion of memory accesses, or a subset of information flows
may be tracked in DIFT for attack detection. In this scenario, the effectiveness of 
partial monitoring can be measured as the percentage of run-time checks that are
performed on each program run, which reflects how likely for a bug or an attack 
to be detected for a system. 

{\bf Profiling:} 
The run-time monitoring system can be used to implement various profiling tools
to collect statistics on program behaviors for performance optimizations as well
as security protection. For example, a recent study showed that an instruction
mix can be used to identify malware from normal programs \cite{FIXME}. 
In such profiling tools, the partial monitoring framework can be used to obtain
the statics on samples rather than all program events, essentially trading off
accuracy for lower overhead.

\subsection{Design Challenges}

While conceptually simple, designing a general framework to dynamically adjust monitoring
overhead introduces new challenges that need to be addressed.  
The following summarizes the main design goals and associated design challenges.

\begin{enumerate}
  \item \textbf{General:} The framework needs to be general enough to be applicable to 
  a wide range of monitoring schemes. Because we target flexible, general-purpose
  run-time monitoring systems, a solution that only works for one monitoring scheme will
  not work.

  \item \textbf{No false positive:} The framework needs to ensure that dropping a portion
  of monitoring operations does not lead to a false positive. We solve this problem using
  a data-flow engine that tracks invalid monitoring meta-data (Section~\ref{sec:dropping}).

  \item \textbf{Intelligent dropping:} The framework needs to match the overhead budget
  while maximizing the effectiveness of monitoring. To this end, adjustable overhead
  monitoring needs to carefully choose which operations to drop and when. We address this
  problem by studying different dropping policies (Section~\ref{sec:policies}) and their
  trade-offs.

\end{enumerate}

