\section{Adjustable Run-Time Monitoring}
\label{sec:monitoring}

% Previous work overheads
\begin{table*}[t]
  \begin{center}
    \vspace{-0.0in}
    \begin{footnotesize}
    %\begin{tabular}{|l|p{2in}|r|r|}
\begin{tabular}{|l|l|l|r|}

\hline
{\bf Name} & {\bf Type} & {\bf Monitoring scheme and flexibility} & {\bf Slowdown (avg./worst)} \\ \hline\hline

DIFT \cite{dift-asplos04} & Custom HW & DIFT only & 1.1\% / 23\% \\ \hline
FlexiTaint \cite{flexitaint-hpca08} & Custom HW & DIFT w/ programmable policies & 1.1\%-3.7\% / 8.7\% \\ \hline
Hardbound \cite{hardbound-asplos08} & Custom HW & Array bounds checks only & 5\%-9\% / 22\% \\ \hline
Harmoni \cite{harmoni-dsn12} & Custom HW & Tag-based monitors & 1\%-10\% / 20\% \\ \hline\hline

FlexCore \cite{flexcore-micro10} & Dedicated FPGA & Instruction-trace monitoring & 1.05x-1.44x / 1.84x \\ \hline
FADE \cite{FIXME} & Core+Custom HW & Instruction-trace monitoring (effective when HW filters work) & \\ \hline
LBA-accelerated \cite{lba-isca08} & Multi-core+Custom HW & Instruction-trace monitoring (effective when accelerators work) & 1.02x-3.27x / 5x \\ \hline
LBA \cite{lba-asid06} & Multi-core+Custom HW & Instruction trace monitoring & 3.23x-7.80x / 11x \\ \hline \hline

Multi-core DIFT \cite{nagarajan-interact08} & SW (multithreaded) & DIFT (compiled for each application) & 1.48x / 2.2x \\ \hline

LIFT \cite{lift-micro06} & SW (DBI) & DIFT (fully flexible) & 3.6x / 7.9x \\ \hline
Purfiy \cite{purify-usenix92} & SW (DBI) & Memory leak checks (fully flexible) & 2.3x / 5.5x \\ \hline
TaintCheck \cite{taintcheck-05} & SW (DBI) & DIFT (fully flexible) & 10x / 27x \\ \hline


\end{tabular}

    \end{footnotesize}
    \caption{Trade-off between performance overhead and flexibility/complexity of run-time monitoring systems.}
    \vspace{-0.2in}
    \label{tab:monitoring.previous_overheads}
  \end{center}
\end{table*}

\subsection{Overhead of Run-Time Monitoring}

There have been a number of proposals for run-time monitoring systems exploring various
design points. % in the trade-off space between efficiency and flexibility.
Table~\ref{tab:monitoring.previous_overheads} summarizes some of representative designs
and their reported performance overhead. The previous studies clearly show that there
exist trade-offs among efficiency, flexibility, and hardware costs. 
For example, a run-time monitoring scheme can often be realized with fairly low
performance overhead (less than 20\%) if implemented with custom hardware that is
designed only for one monitor or a narrow set of monitors. However, the custom
hardware monitors cannot be modified or updated, and require dedicated silicon area. 
One the other hand, flexible systems that support a wide range of monitors lead 
to noticeable performance overhead, often too high for wide deployment in practice.
Software-only implementations \cite{FIXME} or multi-core monitors with minimal
hardware changes \cite{lba-asid06} are reported to have severalfold slowdowns.
On-chip FPGA monitors \cite{flexcore-micro10} and cores with monitoring accelerators
\cite{lba-isca08, fade-hpca14} can reduce overhead significantly, but still show
slowdowns of several tens of percents in some cases.


\subsection{Partial Monitoring for Adjustable Overhead}

In this paper, we investigate adding a new dimension to the trade-off space of 
run-time monitors by allowing coverage or accuracy of a monitor to be traded
off for lower performance and/or hardware overhead.

namely coverage/accuracy of monitors.


will focus our discussion on using a parallel
processing core to implement the monitor. Our evaluation includes results
for both the core-based monitor as well as a higher performance FPGA-based monitor.

Benefits - low overhead or simpler HW

\subsection{Applications and Metrics}

Three main app
- Partial protection for run-time attacks or bugs: you care about the dynamic coverage 
- Soft real-time systems: tight deadline required
- Sampling for Profiling: target metric - accuracy

\subsection{Challenges}

There were three main challenges in designing this dropping hardware:
\begin{enumerate}
  \item \textbf{General:} The hardware needs to be applicable to a wide range of monitoring schemes.
  \item \textbf{No false positives:} False positives should never occur as a result of dropping.
  \item \textbf{Intelligent dropping:} The hardware needs to maximize the amount of useful monitoring done while staying within the overhead budget.
\end{enumerate}



