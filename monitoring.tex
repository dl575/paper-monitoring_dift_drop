\section{Adjustable Run-Time Monitoring}
\label{sec:monitoring}

% Previous work overheads
\begin{table*}[t]
  \begin{center}
    \vspace{-0.0in}
    \begin{footnotesize}
    %\begin{tabular}{|l|p{2in}|r|r|}
\begin{tabular}{|l|l|l|r|}

\hline
{\bf Name} & {\bf Type} & {\bf Monitoring scheme and flexibility} & {\bf Slowdown (avg./worst)} \\ \hline\hline

DIFT \cite{dift-asplos04} & Custom HW & DIFT only & 1.1\% / 23\% \\ \hline
FlexiTaint \cite{flexitaint-hpca08} & Custom HW & DIFT w/ programmable policies & 1.1\%-3.7\% / 8.7\% \\ \hline
Hardbound \cite{hardbound-asplos08} & Custom HW & Array bounds checks only & 5\%-9\% / 22\% \\ \hline
Harmoni \cite{harmoni-dsn12} & Custom HW & Tag-based monitors & 1\%-10\% / 20\% \\ \hline\hline

FlexCore \cite{flexcore-micro10} & Dedicated FPGA & Instruction-trace monitoring & 1.05x-1.44x / 1.84x \\ \hline
FADE \cite{FIXME} & Core+Custom HW & Instruction-trace monitoring (effective when HW filters work) & \\ \hline
LBA-accelerated \cite{lba-isca08} & Multi-core+Custom HW & Instruction-trace monitoring (effective when accelerators work) & 1.02x-3.27x / 5x \\ \hline
LBA \cite{lba-asid06} & Multi-core+Custom HW & Instruction trace monitoring & 3.23x-7.80x / 11x \\ \hline \hline

Multi-core DIFT \cite{nagarajan-interact08} & SW (multithreaded) & DIFT (compiled for each application) & 1.48x / 2.2x \\ \hline

LIFT \cite{lift-micro06} & SW (DBI) & DIFT (fully flexible) & 3.6x / 7.9x \\ \hline
Purfiy \cite{purify-usenix92} & SW (DBI) & Memory leak checks (fully flexible) & 2.3x / 5.5x \\ \hline
TaintCheck \cite{taintcheck-05} & SW (DBI) & DIFT (fully flexible) & 10x / 27x \\ \hline


\end{tabular}

    \end{footnotesize}
    \caption{Trade-off between performance overhead and flexibility/complexity of run-time monitoring systems.}
    \vspace{-0.2in}
    \label{tab:monitoring.previous_overheads}
  \end{center}
\end{table*}

\subsection{Overhead of Run-Time Monitoring}

There have been a number of proposals for run-time monitoring systems, exploring various
design points in the trade-off space between efficiency and flexibility.
Table~\ref{tab:monitoring.previous_overheads} summarizes some of representative designs
and their reported performance overhead.

summarizes some of these implementations and the average and worst-case
overheads reported. These systems vary in their hardware complexity,
flexibility, monitoring schemes, and overheads. Dedicated hardware can provide low 
overheads for specific monitoring schemes. However, more pronounced overheads are seen for 
flexible monitoring systems that implement monitoring on a processor core
\cite{lba-asid06, lba-isca08, nagarajan-interact08} or using an FPGA fabric
\cite{flexcore-micro10}. These systems allow a wide range of monitoring schemes to
be implemented on the same hardware architecture and also allow monitors to be modified and updated after deployment. However, in order to support this flexibility, they can easily incur tens of percent of
overhead in the average-case and severalfold slowdowns in the worst-case. Partial monitoring
can be a useful technique for reducing overheads for any of these systems.
In this paper, we will focus our discussion on using a parallel
processing core to implement the monitor. Our evaluation includes results
for both the core-based monitor as well as a higher performance FPGA-based monitor.

\subsection{Adjustable and Programmable Monitoring}

There were three main challenges in designing this dropping hardware:
\begin{enumerate}
  \item \textbf{General:} The hardware needs to be applicable to a wide range of monitoring schemes.
  \item \textbf{No false positives:} False positives should never occur as a result of dropping.
  \item \textbf{Intelligent dropping:} The hardware needs to maximize the amount of useful monitoring done while staying within the overhead budget.
\end{enumerate}


\subsection{Applications and Metrics}


