\section{Introduction}
\label{sec:intro}

% Parallel run-time useful
Run-time monitoring techniques have been shown to be useful for improving the
reliability, security, and debugging capabilities of computer systems. For
example, Hardbound is a hardware-assisted technique to detect out-of-bound
memory accesses, which can cause undesired behavior or create a security
vulnerability if uncaught \cite{hardbound-asplos08}.  Intel has
recently announced plans to support buffer overflow protection similar to
Hardbound in future architectures \cite{intel-mpx}. Similarly, run-time
monitoring can enable many other new security, reliability, and debugging
capabilities such as fine-grained memory protection \cite{mondrian-asplos02},
information flow tracking \cite{dift-asplos04, testudo-micro08}, hardware error
detection \cite{argus-micro07}, data-race detection \cite{radish-isca12,
cord-hpca06}, etc. 

% Programmable hardware monitors
These run-time monitors introduce performance, power, and energy overheads.
Implementing run-time monitoring in software using binary instrumentation or
other similar methods introduces high overheads. For example,
dynamic information flow tracking (DIFT) implemented in software suffers a 3.6x
slowdown \cite{lift-micro06}. Implementing monitors in hardware greatly decreases
the performance impact by performing monitoring in parallel to a program's
execution. A dedicated hardware implementation of DIFT reduces performance
overheads to just 1.1\% \cite{dift-asplos04}. However, fixed hardware loses
the programmability of software, and cannot be updated or changed.

% Recent studies proposed programmable parallel hardware as a way to achieve both
% the performance advantages of hardware and the flexibility of software
% \cite{lba-isca08, flexcore-micro10, harmoni-dsn12}.  While more
% efficient than software, these programmable hardware monitors can still show
% significant performance overheads.  For example,
% Figure~\ref{fig:intro.full_mon} shows the execution time for three monitoring
% techniques normalized to the execution time with no monitoring for an
% FPGA-based parallel monitor. The program being monitored is run on an in-order
% embedded core running at 500 MHz while the FPGA-based monitor runs at 250 MHz.
% We can see that, depending on the monitoring scheme, overheads can easily be
% several tens of percent.  A faster or more complex (e.g., superscalar or
% out-of-order) core is expected to show even more significant overheads.

% % Run-time monitoring overview
% \begin{figure}
%   \begin{center}
%     \vspace{-0.2in}
%     \includegraphics[width=\columnwidth]{figs/full_mon.pdf}
%     \vspace{-0.5in}
%     \caption{Performance overheads of FPGA-based monitors.}
%     \vspace{-0.2in}
%     \label{fig:intro.full_mon} 
%   \end{center}
% \end{figure}

In this paper, we present an architecture that combines hardware-based dynamic
information flow tracking with a software-based monitor in order to reduce the
overheads of monitoring. By using the DIFT engine to filter out events that do
not correspond to valid metadata, we can greatly reduce the amount of
monitoring that must be handled by the monitor. For example, for an array
bounds check, we are able to filter out instruction that do not correspond to
pointer data. Since the monitor is implemented in software, the system allows a
large range of monitors to be implemented and even updated or changed at
run-time. 

With this filtering, we see overheads of XX\% - XX\% on average for
uninitialized memory check, array bounds check, and multi-bit dynamic
information flow tracking. Although full monitoring coverage is always desired,
if these overheads are considered too high by the designer or user then
monitoring must be disabled.
% % Current schemes are all-or-nothing: suffer full performance impact for full
% % coverage.
% Traditionally, these overheads have been an all-or-nothing cost. If the
% designer finds that the overheads exceed their desired budget, then monitoring
% cannot be enabled.  In this paper, we present an architecture that enables a
% trade-off between the overheads and the coverage of a monitoring technique by
% dynamically adjusting the number of monitoring operations.  This system allows
% monitoring to still be performed even when the overhead budget is not
% sufficient for full monitoring.
% Use cases
Trading off coverage for
reduced overheads can still be very useful. For example, this adjustable
monitoring enables a level of protection even for systems where the full
monitoring overheads are too high.  This is especially true for energy- or
power-constrained systems or soft real-time systems where the monitoring
overhead should not exceed energy/power limits or real-time deadlines. 
Additionally, adjustable monitoring can be used for debugging purposes
especially in the context of sampling-based or cooperative debugging techniques
which expect low coverage per run but use a large number of runs and users to
collect debugging information~\cite{liblit-pldi05,
chilimbi-asplos04, greathouse-cgo11}. 

Thus, we develop an architecture that enables a
trade-off between the overheads and the coverage of a monitoring technique by
dynamically adjusting the number of monitoring operations. By extending the
hardware DIFT engine to track 2-bits of information, we are able to use it for
filtering empty metadata as well as tracking dropped monitoring flows in order
to prevent false positives. We introduce three policies of where dropping
decisions are made and show the trade-off space achieved between accuracy of
overhead matching and coverage achieved.

% %Invalidation in order to prevent false positives.
% In order to adjust monitoring overheads, a system must be able to drop a
% portion of the monitoring operations. Unfortunately, however, simply skipping
% monitoring operations can lead to false positives (false alarms) due to stale
% metadata.  One possible solution is to rewrite individual monitors to create
% ``drop-enabled'' versions.  However, such monitor-specific customizations are
% difficult and time consuming.  Instead, after analyzing a range of monitoring
% techniques, we found that by adding a 1-bit ``invalid'' flag to the bookkeeping
% metadata managed by the monitor, we were able to create a general mechanism to
% handle false positives across a broad range of monitoring techniques.  The
% result is that we have designed a hardware module that sits between a
% processing core and a programmable monitor, and allows adjustable overheads.
% This hardware module can be applied to a range of monitoring techniques and
% requires no modifications to the monitor.

% Evaluation
In order to evaluate our approach, we applied it to 3 different monitoring
techniques. These monitoring techniques vary in what events they monitor, the
size and semantics of their metadata, and the operations performed on metadata.
Our experiments show that with 10\% overheads, we can still provide 49-90\% of
coverage on average depending on the monitoring technique. By increasing the
overhead budget, the coverage rate can be increased. Our architecture is able
to closely meet the specified budgets. For all but one benchmark, the overheads
seen were within 2\% of the target overhead. Our results show that X policy is
able to achieve very close overhead matching (within X\% on average) with an
average coverage of X\%. On the other hand, by using Y policy, the overhead
matching is not as high (within Y\% on average) but average coverage increases
to (Y\%).

% Section summary
This paper is organized as follows. Section~\ref{sec:monitoring} presents the
parallel run-time monitoring model that this paper assumes.
Section~\ref{sec:filter} presents our architecture that use hardware-based DIFT
in order to reduce monitoring overheads. In Section~\ref{sec:drop} we present
how the DIFT engine can be extended to enable adjustable overheads.
Section~\ref{sec:extensions} covers example implementations of three monitoring
schemes. Section~\ref{sec:evaluation} presents our evaluation methodology and
results. Finally, we discuss related work in Section~\ref{sec:related} and
conclude in Section~\ref{sec:conclusion}.

