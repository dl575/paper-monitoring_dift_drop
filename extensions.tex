\section{Application to Monitoring Schemes}
\label{sec:extensions}

% Monitoring operations
\begin{table*}[tb]
  \begin{center}
    \begin{small}
    
% Summary of monitoring operations

\begin{tabular}{|l|l|l|}
\hline

{\bf Monitor} & {\bf Event} & {\bf Monitoring Operation} \\ \hline \hline

\multirow{4}{*}{BC}  
 & Array allocation & Set base and bound metadata \\ \cline{2-3}
 & ALU & Copy metadata from register to register \\  \cline{2-3}
 & Load & Copy metadata from memory to register. Check memory access address. \\ \cline{2-3}
 & Store & Copy metadata from register to memory. Check memory access address. \\ 
\hline\hline

\multirow{2}{*}{UMC}  
 & Store & Set metadata tag (data initialized) \\ \cline{2-3}
 & Load & Check if metadata tag is set \\ 
\hline\hline
% & Software clear tag & Clear metadata tag & Non-critical \\ \hline\hline

\multirow{5}{*}{DIFT}  
 & Software set tag & Set tag to a specified identifier value based on software command \\ \cline{2-3}
 & ALU & If input tags are different, output tag is a new identifier. Otherwise, output tag is the input tag. \\ \cline{2-3}
 & Load & Set register tag as tag of loaded data \\ \cline{2-3}
 & Store & Set memory tag as register tag \\ \cline{2-3}
% & Software clear tag & Clear tag & Update \\ \cline{2-4}
 & Indirect jump & Check tag \\ 
 \hline

% \multirow{5}{*}{BC}  
%  & MOV/LD/ST          & Propagate colors (Tag[dest] = Tag[src]) & Critical \\ \cline{2-4}
%  & ADD/SUB/NOT        & Update colors (Tag[dest] = f(Tag[src])) & Critical \\ \cline{2-4}
%  & OR/XOR             & Clear colors & Critical \\ \cline{2-4}
%  & LD/ST              & Compare pointer and memory colors & Check \\ \cline{2-4}
%  & Software set color & Set pointer colors & Critical \\ \hline\hline

% Soft Error Check & Instruction & Check instruction computation & Check \\ \hline\hline
% 
% \multirow{2}{*}{Bank of Observers}  
%  & System output & Update state estimation & Critical \\ \cline{2-4}
%  & New state estimate & Check that states of all observers matches & Check \\ \hline

\end{tabular}

    \end{small}
    \caption{Monitoring operations for BC, UMC, and DIFT.}
    \label{tab:extensions.monitor}
    \vspace{-0.2in}
  \end{center}
\end{table*}

% Filtering operations for clean flags
\begin{table*}[tb]
  \begin{center}
    \begin{small}
    
% Operations for handling clean flags

\begin{tabular}{|l|l|l|l|}
\hline

{\bf Monitor} & {\bf Event} & {\bf Filter Condition} & {\bf Filter Operation} \\ \hline \hline

\multirow{4}{*}{BC}  
& Array allocation & Never & - \\ \cline{2-4}
& ALU   & Both source registers are clean & Mark destination register as clean \\ \cline{2-4}
& Load  & Source memory location is clean & Mark destination register as clean \\ \cline{2-4}
& Store & Source register is clean & Mark destination memory location as clean \\
\hline\hline

\multirow{2}{*}{UMC}  
& Store & Never & - \\ \cline{2-4}
& Load & Source memory location is dirty & Do nothing \\ 
\hline\hline

\multirow{5}{*}{DIFT}  
& Software set tag & Never & - \\ \cline{2-4}
& ALU   & Both source registers are clean & Mark destination register as clean \\ \cline{2-4}
& Load  & Source memory location is clean & Mark destination register as clean \\ \cline{2-4}
& Store & Source register is clean & Mark destination memory location as clean \\ \cline{2-4}
& Indirect jump & Source register is clean & Do nothing \\ 
\hline


\end{tabular}

    \end{small}
    \caption{Filtering conditions and operations for clean flags.}
    \label{tab:extensions.filter}
    \vspace{-0.2in}
  \end{center}
\end{table*}

% Filtering operations for invalid flags
\begin{table*}[tb]
  \begin{center}
    \begin{small}
    
% Operation of hardware modules

\begin{tabular}{|l|l|l|l|}
\hline

{\bf Monitor} & {\bf Event} & {\bf Filter Condition} & {\bf Dropping/Filter Operation} \\ \hline \hline

\multirow{4}{*}{BC}  
& Array allocation & Never & Invalidate array pointer \\ \cline{2-4}
& ALU   & At least one invalid source register & Invalidate destination register \\ \cline{2-4}
& Load  & Invalid source memory location & Invalidate destination register \\ \cline{2-4}
& Store & Invalid source register & Invalidate destination memory location \\
\hline\hline

\multirow{2}{*}{UMC}  
& Store & Never & Invalidate destination memory location \\ \cline{2-4}
& Load & Invalid source memory location & Do nothing \\ 
\hline\hline

\multirow{5}{*}{DIFT}  
& Software set tag & Never & Invalidate destination memory location \\ \cline{2-4}
& ALU   & At least one invalid source register & Invalidate destination register \\ \cline{2-4}
& Load  & Invalid source memory location & Invalidate destination register \\ \cline{2-4}
& Store & Invalid source register & Invalidate destination memory location \\ \cline{2-4}
& Indirect jump & Invalid source metadata & Do nothing \\ 
\hline


\end{tabular}

    \end{small}
    \caption{Filtering conditions and operations performed for a dropped or filtered event for invalid flags.}
    \label{tab:extensions.drop}
    \vspace{-0.2in}
  \end{center}
\end{table*}

In this section, we show examples of how our architecture can be applied to
three specific monitoring schemes: array bounds check (BC), uninitialized
memory check (UMC), and dynamic information flow tracking (DIFT).
Table~\ref{tab:extensions.monitor} shows a summary of the monitoring operations for these
monitoring schemes. Table~\ref{tab:extensions.filter} summarizes when when
monitoring events can be filtered out due to clean metadata and what metadata
needs to be marked as clean on a filtered event.
Table~\ref{tab:extensions.drop} shows the analogous information for the valid
flags. In this case, the operation listed is also the one performed when a
monitoring event is dropped due to insufficient slack. Note that for simplicity
the policies for clean and invalid flags are shown and discussed separately
here. However, the dataflow engine actually operates on pairs of these flags
using a combined policy.

\subsection{Array Bounds Check}

As described in Section~\ref{sec:monitoring}, array bounds check is a
monitoring scheme that attempts to detect buffer overflows where memory
accesses go beyond the boundaries of an array.
Buffer overflows can cause unreliable program execution or create openings for
malicious attackers. One way to detect buffer overflows is a base and bounds
approach \cite{hardbound-asplos08}. In this approach, metadata is
associated with an array pointer when it is allocated that includes the base
(start) address and bound (end) address of the corresponding array. On ALU
instructions, this base and bound metadata is copied to the destination
registers.  On store instructions, the metadata for the source register is
copied to the metadata for the store address. Similarly, on load instructions,
the metadata value for the load address is copied to the destination register's
metadata. In addition, on load and store instructions, the memory address is
checked to be within the base and bounds metadata. If it is not, then an
out-of-bounds error is detected. These operations are summarized in
Table~\ref{tab:extensions.monitor}.

In terms of filtering using clean flags (Table~\ref{tab:extensions.filter}),
array allocations are never filtered since these will always involve non-clean
metadata (i.e., setting new base and bounds). For ALU operations, if both
source operands are clean, then the event can be filtered by setting the
destination register as clean. For load and store, only a single source needs
to be checked to be clean in order for the event to be filtered. Similarly, the
destination register or memory location needs to be marked as clean if the
event is filtered.

The invalid flags for BC operate similarly to the clean flags
(Table~\ref{tab:extensions.drop}). When ALU, load, or store events are dropped or filtered,
the destination register or memory location is marked as invalid. In addition,
we need to handle the possibility of dropping an array allocation. In this
case, the array pointer is marked as invalid. In terms of filtering out invalid
events, for an ALU event, if either of the source registers is invalid then the
event can be filtered. For load and store instructions, if the single source
operand is invalid, then the event can be filtered.

\subsection{Uninitialized Memory Check}

Uninitialized memory check is a monitoring scheme that checks that memory is
written to before it is read from. This is done by forwarding all load and
store instructions to the monitoring core. For every byte of memory, the
monitor keeps one bit of metadata. On a store to a memory location, the monitor
marks the corresponding metadata bit to indicate that the memory location has
been initialized. On a load, the monitor checks that the corresponding metadata
bit has been previously marked as initialized.

Rather than using the clean flags to mirror the initialization metadata that
UMC stores, we use the clean flags as a coarse-grained version of the UMC
metadata in order to more efficiently filter out monitoring operations. We use
a clean flag per word (four bytes) of memory. A clean flag is set to dirty only
if all four of the corresponding memory bytes have been marked as initialized.
On a load, if the clean flag is marked as dirty, then we can filter the event
since we know that all possible corresponding location memory locations have
been marked as initialized. If the flag is marked as clean, then that indicates
that at least one of the corresponding memory bytes has not been initialized
and the monitoring core must perform a further check. Store instructions are
never filtered out.

On a dropped store instruction, the destination memory location is marked as
invalid. On load instructions, we can filter them out if the source memory
location has been invalidated. In this case, we do not have to perform any
further operation since on load instructions since UMC only performs a check
operation for loads.

\subsection{Dynamic Information Flow Tracking}

Dynamic information flow tracking (DIFT) \cite{dift-asplos04} is a security
monitoring scheme that seeks to detect when information from untrusted sources
is used to affect the control flow.  In its simplest form, DIFT keeps a 1-bit
metadata tag for each memory location and each register, indicating tainted or
untainted. When data is read from an untrusted source, it is marked as tainted.
As this data is used for other operations, the results of these operations are
also marked as tainted. On an indirect jump, the taint bit of the register used
is checked. If the register is found to be tainted, then an error is detected.
We focus on a more heavy-weight approach that also tracks some information of
where flows originate from by using a multi-bit (32-bits in our implementation)
metadata tag. This can be used to assist debugging in the case of an error. In
this case, when data is marked from an untrusted source, the software specifies
an identifier to set the metadata tag. This multi-bit tag is propagated on
load, store, and ALU operations. If two different tags are used as sources to
an ALU operation, then a new identifier is calculated based on the sources.
These monitoring operations are summarized in
Table~\ref{tab:extensions.monitor}.

Table~\ref{tab:extensions.filter} shows how the clean flags are used for DIFT.
The rules for ALU, load, and store instructions are same as for array bounds
check. Similar to array allocations, for DIFT, software setting of metadata
tags is never filtered out due to clean metadata. On indirect jump
instructions, if the source register is clean, then the event can be filtered
out as no error has occurred.

Again, the invalid flags for DIFT operate similar to their operation for BC
(Table~\ref{tab:extensions.drop}). In addition to operations for invalidating
destination operands for ALU, load, and store instructions, the destination
memory location is also invalidated when set tag operations dropped. Indirect
jump events are filtered out if the source register is marked as invalid.
However, no further metadata needs to be marked as invalid in this case.

\subsection{Applicability to Other Schemes}

In this section, we have discussed how our dataflow engine can be applied to
three example monitoring schemes to enable reduced and adjustable overheads.
However, our approach can be generally applied to many monitoring schemes. This
is possible because the use of clean and invalidation flags is largely agnostic
to the semantics of the specific monitoring scheme. That is, the dataflow
engine must be configured based on the dataflow of the monitoring scheme, but
it does not need detailed information of what the monitoring scheme does. 

We do note some limitations to the proposed approach. Previous work
\cite{fade-hpca14} that has looked at filtering in order to reduce monitoring
overheads has also filtered out redundant updates where the source metadata
matches the destination metadata. Since we do not read the actual metadata, our
architecture is not able to identify these redundant updates. However, by using
only 2-bit flags, the bandwidth need of our dataflow engine is reduced.  
In addition, we note that filtering using clean flags works well when a large
portion of instructions are not relevant to the monitoring scheme (e.g., array
bounds check only cares about array pointers). However, if most instruction
have dirty metadata, then very little filtering will be possible. 

In terms of using invalidation to perform partial monitoring, this 
approach works best for monitoring schemes where there are many pieces of
independent metadata. Since marking a piece of metadata as invalid will cause
future metadata with dependences to also be marked as invalid, schemes with
large metadata dependency chains can quickly see large amounts of metadata
become invalid. 

Another limitation is that the dataflow engine, as described, only marks one
flag location on each monitoring event. Similarly, the dataflow engine can only
read in two sets of flags in order to determine whether an event can be
filtered or not. This was done because we found that for many monitoring
schemes, the monitoring operations corresponded to program instructions, which
read up to two operands and write to one destination.  For monitoring schemes
which update multiple pieces of metadata on a monitoring event, it may be
adequate to associate a clean and invalid flag with a single piece of metadata.  For
example, a scheme may use a data structure that includes several pieces of
related metadata. On a drop, one of these pieces could be marked as invalid in
order to indicate that the entire structure is invalid.  Similarly, marking one
specified piece as clean may indicate that the entire data structure's metadata
is clean.  Alternatively, the read and write capabilities of the dataflow
engine could be expanded to support more complex monitoring schemes.
